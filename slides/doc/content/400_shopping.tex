%!TEX root = ../document.tex
\section{Compras}
O passo seguinte a promover o hábito de visita é, naturalmente, incentivar o utilizador a adquirir produtos. O mecanismo sugerido para o efeito consiste em colecções, e baseia-se nas missões descritas na secção anterior.

Colecções consistem em conjuntos de objectos com alguma forma de relação.
Quando o utilizador inicia a compra de um dado produto, se este se inserir numa dada colecção, é dada a hipótese ao utilizador de adquirir mais produtos da mesma. Isto seria motivado pela atribuição de descontos ou créditos na plataforma, cuja quantia estaria directamente relacionada com quantidade de produtos da colecção adquiridos e o respectivo valor (colecções mais caras valem mais).

Estas colecções seriam adaptadas a cada utilizador de acordo com o seu histórico de compras, de modo a maximizar a probabilidade de adquirir todo o conjunto de uma vez. Complementarmente, também é interessante a criação de colecções alusivas a festividades (Natal, Halloween, etc).

De salientar que, existindo a possibilidade de criar colecções correspondentes a quantias avultadas, o utilizador deveria ter a possibilidade de adquirir os produtos em falta na colecção em compras separadas, sendo apenas penalizado de forma proporcional ao intervalo de tempo decorrido. Assim mantém-se a motivação para que complete a colecção, recompensando aqueles que o fazem mais rapidamente.