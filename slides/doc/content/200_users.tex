%!TEX root = ../document.tex
\section{Divulgação}
Sendo o Plazr um projecto cujo propósito final é tirar partido do público consumidor online e intermediar transacções entre utilizadores da web e vendedores sem representação nesse meio, é natural que a divulgação da plataforma seja uma prioridade.
A presença de mais utilizadores automaticamente se reflecte numa maior probabilidade de consumo e maior visibilidade para os clientes.
Adicionalmente, a qualquer momento um utilizador pode tornar-se um cliente.
A existência de uma quantidade considerável de utilizadores pode ainda traduzir-se em maiores margens de lucro por exploração de meios publicitários integrados nas páginas da plataforma.

A forma mais eficaz, e consequentemente mais utilizada, para a divulgação deste tipo de projectos resume-se a recorrer à rede social de cada utilizador já registado para alcançar outros possíveis -- referências.
Apesar de ser uma estratégia muito comum, os resultados estão directamente relacionados com a predisposição do utilizador para investir algum do seu tempo a referir os elementos da sua rede.
Como tal, é possível aumentar consideravelmente a eficácia deste mecanismo providenciando a motivação adequada ao utilizador.

A motivação principal para qualquer utilizador na plataforma, enquanto possível consumidor, é a possibilidade de obter crédito em compras futuras. Assim, sugere-se que uma forma básica de incentivar a referência de novos utilizadores será concedendo ao referenciador crédito no valor absoluto do lucro obtido com a primeira compra do utilizador referenciado. O facto de se utilizar o valor absoluto torna isto independente da forma como a plataforma obtém lucro com cada loja, nunca resultando em prejuízo directo.

Também como motivação, a estipulação de níveis, onde um utilizador seria recompensado por referenciar um dado número de novos utilizadores, permite variar além da recompensa em crédito, sendo uma oportunidade para reconhecer a conquista ou oferecer bens virtuais. O primeiro caso implicaria patamares de dificuldade crescente, enquanto que o segundo permitiria ao utilizador obter recompensas com maior frequência.
