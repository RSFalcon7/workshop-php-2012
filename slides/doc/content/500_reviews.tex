%!TEX root = ../document.tex
\section{Opiniões}
Finalmente, uma das questões mais difíceis no Plazr relaciona-se com a necessidade de obter avaliações dos utilizadores relativamente às lojas em que efectuaram compras. Numa tentativa de dar resposta sugere-se, mais uma vez, a integração de missões, agora de carácter mais frequente.

Após um dado utilizador efectuar uma compra, o sistema cria uma missão para que este submeta uma avaliação do produto comprado bem como uma da respectiva loja. Apesar de não existir qualquer impedimento, esta missão não deve ser criada logo após a compra, prevendo ou estimando o tempo de entrega do produto, de forma a evitar que o utilizador submeta uma avaliação infundada.
Esta acção seria recompensada com representações de conquista, em particular com estatutos (ou títulos), que dariam aos utilizadores alternativas alusivas ao Plazr na interacção com a sua rede social.

Complementarmente, sendo que a qualidade das avaliações também representa um problema, sugere-se a implementação de um mecanismo de \textit{peer assessment} baseado em missões curtas e frequentes (carácter diário por exemplo). Assumindo que, na maioria, os utilizadores se manteriam objectivos nesta avaliação, isto permitiria elevar a qualidade das missões de forma geral. Estatutos (ou outras representações de conquista) ou bens virtuais seriam a melhor forma de motivar os utilizadores a completarem estas missões com alguma regularidade. É importante notar que este tipo de avaliações teriam de ser condicionadas, pelo menos em termos de tamanho, de forma a que os utilizadores se mantivessem motivados para realizar as missões em troca de recompensas mínimas.

Este último mecanismo, ainda que complementar, teria a vantagem de influenciar directamente os utilizadores para desenvolverem o hábito de visitar a plataforma regularmente.