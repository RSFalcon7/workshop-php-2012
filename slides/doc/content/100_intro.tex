%!TEX root = ../document.tex
\section{Introdução}
A integração de mecanismos e técnicas de desenvolvimento de jogos em modelos de negócio é, hoje em dia, bastante comum.
Consiste em repensar estratégias de marketing do ponto de vista de um criador de jogos, de forma a proporcionar aos clientes soluções que se revelam, simultaneamente, atrativas e divertidas.
Isto permite ultrapassar algumas barreiras % inerentes aos novos negócios, tais como a divulgação ou a conquista de uma quota de mercado.
e transformar a experiência do utilizador, de modo a torná-la mais cativante.

No caso particular do Plazr, apesar dos clientes directos preferirem profissionalismo sério que lhes transmita garantias, tal não é o caso dos utilizadores (os compradores finais).
Estes são, na generalidade, receptivos à integração de elementos geralmente associados a jogos nas ferramentas interactivas a que recorrem no quotidiano.
Assumindo a presença de utilizadores mais participativos nos meios online, o que está geralmente associado a um grupo demográfico mais jovem, a sua familiariedade leva a que elementos directamente relacionados com a mecânica de jogos devam ser considerados, mesmo que este grupo não represente a maior quota do mercado alvo. Estes elementos não implicam custos directos no modelo de negócio e permitem cativar alguns nichos dentro do mercado online.

Entre as mecânicas viáveis para o Plazr em concreto, destacam-se:
\begin{itemize}
\item Elementos representadores de conquistas (\textit{badges}, \textit{achievements}, títulos e níveis, por exemplo), entregues a um utilizador quando este completa determinados objectivos ou atinge patamares previamente estipulados permitem adicionar à plataforma factores como a competição (ou cooperação) entre utilizadores, a progressão (ou evolução);
\item Bens virtuais, que se diferenciam das representações de conquista por serem mais versáteis, permitirem a inclusão na plataforma de características completamente alheias às lojas (personalização de avatares, por exemplo), e poderem ter um carácter temporal (e.g., festividades tradicionais).
\end{itemize}

Nas subsecções seguintes apresentam-se alguns mecanismos pensados para auxiliar esta plataforma com algumas questões cruciais relacionadas com a interacção com o utilizador.