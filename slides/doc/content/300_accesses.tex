%!TEX root = ../document.tex
\section{Visitas}

Após a divulgação, é necessário manter o utilizador (e potencial comprador) atraído e interessado na plataforma. Em particular, é interessante para o Plazr que os utilizadores desenvolvam o hábito de visita, quer a um conjunto de lojas da sua preferência, quer às páginas centrais nas quais estariam expostos a publicidade e poderiam descobrir novas lojas.
Igualmente, é também importante incentivar os utilizadores a passarem mais tempo nas lojas, aumentando a probabilidade de efectuarem compras.

Para este efeito sugere-se a implementação de missões. Mais concretamente, estas missões poderiam consistir em ``Caças ao tesouro'', nas quais o utilizador receberia uma lista de classes (ou tipos) de produtos e teria de encontrar um produto para satisfazer cada entrada da lista. A utilização de classes no lugar de produtos específicos é importante, pois permite que o objectivo seja independente de qualquer loja específica (diferentes utilizadores podem completar o mesmo objectivo com produtos distintos de lojas distintas).
É importante destacar que não seria necessário o utilizador adquirir o produto, apenas encontrá-lo, motivando-o assim a participar no jogo sem qualquer prejuízo. Note-se ainda que estas missões seriam facilmente extensíveis a um grupo de utilizadores, favorecendo a cooperação.

A implementação de missões na plataforma é uma oportunidade para adicionar factores de progressão e competição em que os utilizadores tentariam completar mais missões de dificuldade crescente. As recompensas para estas poderiam incluir crédito, mas é mais rentável para o Plazr não incorrer em prejuízo directo para recompensar interacções motivadas por outras acções que não o consumo. Como tal, é preferível recompensar os utilizadores com representações de conquista ou bens virtuais.

Adicionalmente, uma motivação óbvia para que os utilizadores efectuem visitas regulares seria a recompensa por visitas consecutivas dentro de intervalos de tempo curtos (visitas diárias, por exemplo). Isto poderia ser facilmente implementado com recurso a um sistema de pontos que os utilizadores poderiam gastar em bens virtuais. Estes pontos são ainda um excelente meio de adicionar representações de conquista baseadas na regularidade das visitas.